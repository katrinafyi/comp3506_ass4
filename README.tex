\documentclass[11pt,a4paper]{article} % Default style and margins
\usepackage{courier}
\input{../../header.tex}
\input{../../problem_set.tex}

\titleformat{\part}
{\normalfont\Large}{Problem }{0 pt}{}[\hrule]

\author{s4529458}
\date{{Due 27/09/2019 5:00 pm}}
\title{COMP3506 -- Assignment 4}

\begin{document}
%\includepdf[pages=1]{Coversheet_45294583_10112906_20190329165713[1522].pdfz}
%\blankpage

\setcounter{page}{1}
\maketitle

\part*{Implementation Details}
\section*{FeedAnalyser Constructor}
This makes use of a HashMap which maps users to a tree map.
HashMap is implemented by the Java Collections Framework using a
hash table \cite{hashmap}. By default (which we use), this has a load factor of $0.75$,
meaning the hash-table is resized when 75\% of its buckets are full.
This is a compromise between time and space costs and results in 
(amortised) $\mathcal O(1)$ insertions and lookups, with linear space usage.
In the worst case, it is $\mathcal O(n)$ if the hash table needs to be 
resized.

The TreeMap maps dates to an ArrayList of posts made on that day.
This is implemented by the JCF as a red-black tree \cite{treemap}. 
This has the property of 
$\mathcal O(\log n)$ insertion and lookup in all cases \cite{clrs}.

Finally, the ArrayList is an array-backed list with constant-time insertions \cite{arraylist}.

Suppose there are $n$ FeedItems and the get methods on FeedItem are $\mathcal O(1)$.
In the constructor, the \verb|while| loop iterates $n$ times, 
each iteration taking $\mathcal O(\log n)$ time because of the 
TreeMap inseration. The other operations (ArrayList add) are $\mathcal O(1)$.
As a whole, this loop takes $\mathcal O(n \log n)$ and the array sorting algorithm used by 
Java is bounded by $\mathcal O(n \log n)$ \cite{sort}.
Thus, the constructor is bounded by
$\mathcal O(n \log n)$ in the worst case.

\section*{getPostsBetweenDate}
This performs a lookup on the HashMap to get one user's posts, 
in $\mathcal O(1)$ time.
Then, we index the nearest index using \verb|subMap()|, \verb|headMap()|
or \verb|tailMap()| to get the range of posts between the given dates. 
This is done using lookups which are always $\mathcal O(\log n)$ 
for a red-black tree since they are balanced \cite{redblack}. Note that 
this TreeMap only contains the posts for this user, so will often contain 
less than $n$ items if there are multiple users posting.

Then, we collect the lists of across all dates in the range into an ArrayList, which takes 
$\mathcal O(k)$ time where $k$ is the number of posts falling within the range.

The algorithm is worst-case $\mathcal O(\log n)$ or $\mathcal O(k)$, whichever is larger,
and usees $\mathcal O(n)$ space as every FeedItem needs to be stored once.

This algorithm is obviously better than brute-force search which 
would take $\mathcal O(n)$ time. An alternative implementation could be using a 
binary search on an array of posts sorted by date. This has the advantage of 
the data structure being simpler than a TreeMap, but operations require manually 
binary searching through the array. We choose TreeMap which provides 
a simpler interface with equivalent $\mathcal O(\log n)$ performance for lookups,
which is advantageous if the code needs to be easy to read and understand,
e.g. for lecture examples.

A disadvantage of this is needing to collect posts from different dates into one 
list to return, which takes $\mathcal O(k)$ time. We assume that $k \ll \log n$ 
so this is not a problem.


\section*{getPostAfterDate}
This performs one lookup on a HashMap and one loopup on a TreeMap. 
These are $\mathcal O(1)$ and $\mathcal O(\log n)$ respectively
(with the same caveats as above). This returns an array
which is indexed in $\mathcal O(1)$ time.
Thus, this is $\mathcal O(\log n)$ worst-case.

\section*{getHighestUpvote}
This is just an array indexing which is  $\mathcal O(1)$ time,
using a precomputed sorted list of posts. An integer variable keeps track of 
which index of the array we are currently up to. This is also incremented each time 
the function is called.

This implementation takes $\mathcal O(n)$ space to store a sorted list, 
where $n$ is the number of total 
posts and is constant time. If space is not a concern, this is the best possible 
result (asymptotically). 

\section*{getPostsWithText}
This is an implementation of the Boyer-Moore algorithm to search 
the text of every post. Suppose $m$ is the length of the  pattern,
 $n$ is the number of posts and $k$ is the maximum text length.

Preprocessing for the Boyer-Moore algorithm takes $\mathcal O(m)$ time
and this is done once for all texts.
The algorithm runs in $\mathcal O(m + k)$ if the 
search pattern does not appear and $\mathcal O(mk)$ if it does \cite{sustik}. 

Thus, searching every post for this pattern will take $\mathcal O(np(mk) + n(1-p)(m+k))$
worst-case where $p$ is the proportion of posts which match the search pattern.
This uses $\mathcal O(m)$ extra space.

An alternative would be to use a suffix tree, which precomputes all suffixes of 
each text. This would increase the preprocessing time (in the constructor) 
to $\mathcal O(nk)$ but decrease the runtime of \verb|getPostsWithText|
to $\mathcal O(nm)$ which, notably, does not depend on text length $k$ \cite{suffix}.
However, this is only advantageous if many substrings will be searched. The current 
Boyer-Moore moethod will be better if few patterns are searched.
 

\bibliography{bibliography}
\bibliographystyle{IEEEtran}

\end{document}
